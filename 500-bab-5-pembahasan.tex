\chapter*{BAB V \\ HASIL DAN PEMBAHASAN}
\addcontentsline{toc}{chapter}{BAB V HASIL DAN PEMBAHASAN}

\setcounter{chapter}{5} 
\setcounter{section}{0} 
\setcounter{figure}{0} 
\setcounter{table}{0}

Bab ini menyajikan hasil pelaksanaan kegiatan penyusunan dan penataan Standar
Operasional Prosedur (SOP) di lingkungan Pemerintah Kabupaten Cianjur serta
pembahasan atas capaian yang telah diperoleh. Penyajian hasil difokuskan pada
output SOP yang telah disusun sebagai pedoman kerja operasional bagi perangkat
daerah, sesuai dengan tugas dan fungsi masing-masing.

Hasil penyusunan SOP yang disajikan dalam bab ini merupakan akumulasi dari
seluruh tahapan kegiatan yang telah dilaksanakan, mulai dari persiapan,
pengembangan, penerapan, hingga monitoring awal. Seluruh proses tersebut
dilakukan dengan mengacu pada ketentuan peraturan perundang-undangan yang
berlaku, khususnya pedoman penyusunan SOP Administrasi Pemerintahan, serta
disesuaikan dengan kebutuhan dan karakteristik organisasi perangkat daerah di
lingkungan Pemerintah Kabupaten Cianjur.

Pembahasan dalam bab ini bertujuan untuk memberikan gambaran yang jelas dan
komprehensif mengenai kondisi akhir SOP yang telah disusun, baik dari sisi jumlah,
cakupan, maupun kesiapan penerapannya. Selain itu, bab ini juga menguraikan
bagaimana hasil penyusunan SOP tersebut mendukung peningkatan tertib
administrasi, keseragaman prosedur kerja, serta penguatan akuntabilitas
penyelenggaraan pemerintahan dan pelayanan publik.

Sebagai bagian dari pertanggungjawaban pelaksanaan kegiatan, hasil penyusunan
dan penataan SOP disajikan secara sistematis agar dapat digunakan sebagai bahan
evaluasi, referensi implementasi, serta dasar pengambilan kebijakan lanjutan oleh
Pemerintah Kabupaten Cianjur. Dengan demikian, pembahasan pada bab ini
diharapkan dapat memberikan nilai tambah dalam memastikan bahwa SOP yang
telah disusun benar-benar siap digunakan dan memberikan manfaat nyata bagi
peningkatan kinerja organisasi perangkat daerah.

\section{Rekapitulasi Standar Operasional Prosedur per Instansi}

RRekapitulasi Standar Operasional Prosedur (SOP) disusun untuk memberikan gambaran
komprehensif mengenai hasil akhir kegiatan penyusunan dan penataan SOP di
lingkungan Pemerintah Kabupaten Cianjur. Rekapitulasi ini menjadi bagian penting
dalam laporan akhir karena menunjukkan capaian kegiatan secara kuantitatif,
baik dari sisi sebaran SOP pada masing-masing instansi maupun dari sisi status
verifikasi SOP dalam sistem.Rekapitulasi Standar Operasional Prosedur (SOP) disusun untuk memberikan gambaran
komprehensif mengenai hasil akhir kegiatan penyusunan dan penataan SOP di
lingkungan Pemerintah Kabupaten Cianjur. Rekapitulasi ini menjadi bagian penting
dalam laporan akhir karena menunjukkan capaian kegiatan secara kuantitatif,
baik dari sisi sebaran SOP pada masing-masing instansi maupun dari sisi status
verifikasi SOP dalam sistem.

\subsection{Rekapitulasi SOP per Instansi}

Rekapitulasi SOP per instansi menggambarkan jumlah SOP yang telah disusun oleh
masing-masing perangkat daerah, kecamatan, dan unit layanan kesehatan. Data ini
menunjukkan tingkat partisipasi dan capaian penyusunan SOP pada setiap instansi,
serta memberikan gambaran mengenai variasi jumlah SOP sesuai dengan tugas dan
fungsi organisasi.

{\footnotesize
\begin{longtable}{|
    >{\centering\arraybackslash}p{1cm}|
    p{10.3cm}|
    >{\centering\arraybackslash}p{2cm}|
}
\caption{Rekapitulasi Jumlah SOP per Instansi} \label{tab:rekap_jumlah_sop} \\
\hline
\rowcolor{lightgray}\textbf{No} & \textbf{Instansi} & \textbf{Jumlah SOP} \\
\hline
\endfirsthead

\hline
\rowcolor{lightgray}\textbf{No} & \textbf{Instansi} & \textbf{Jumlah SOP} \\
\hline
\endhead

\hline
\endfoot

\hline
\endlastfoot

1  & Badan Perencanaan Pembangunan, Riset dan Inovasi Daerah & 6  \\ \hline
2  & Dinas Kependudukan dan Pencatatan Sipil & 28 \\ \hline
3  & RSUD dan Puskesmas & 64 \\ \hline
4  & Kecamatan & 2 \\ \hline
5  & Sekretariat DPRD & 65 \\ \hline
6  & Badan Kepegawaian dan Pengembangan Sumber Daya Manusia & 14 \\ \hline
7  & Inspektorat Daerah & 1 \\ \hline
8  & Bagian Pemerintahan Sekretariat Daerah & 0 \\ \hline
9  & Bagian Kesejahteraan Rakyat Sekretariat Daerah & 0 \\ \hline
10 & Bagian Hukum Sekretariat Daerah & 0 \\ \hline
11 & Bagian Administrasi Pembangunan Sekretariat Daerah & 0 \\ \hline
12 & Bagian Perekonomian dan Sumber Daya Alam Sekretariat Daerah & 0 \\ \hline
13 & Bagian PBJ Sekretariat Daerah & 0 \\ \hline
14 & Bagian Umum dan Keuangan Sekretariat Daerah & 1 \\ \hline
15 & Bagian Organisasi Sekretariat Daerah & 0 \\ \hline
16 & Bagian Protokol dan Komunikasi Pimpinan Sekretariat Daerah & 0 \\ \hline
17 & Dinas Komunikasi, Informatika dan Persandian & 8 \\ \hline
18 & Badan Pendapatan Daerah & 8 \\ \hline
19 & Badan Keuangan dan Aset Daerah & 26 \\ \hline
20 & Dinas Arsip dan Perpustakaan & 23 \\ \hline
21 & Dinas Kesehatan & 46 \\ \hline
22 & Dinas Pendidikan, Pemuda dan Olahraga & 1 \\ \hline
23 & Dinas Kebudayaan dan Pariwisata & 2 \\ \hline
24 & Badan Kesatuan Bangsa dan Politik & 2 \\ \hline
25 & Dinas Pengendalian Penduduk, Keluarga Berencana, Pemberdayaan Perempuan, dan Perlindungan Anak & 38 \\ \hline
26 & Satuan Polisi Pamong Praja dan Pemadam Kebakaran & 5 \\ \hline
27 & Dinas Pertanian, Hortikultura, dan Perkebunan & 0 \\ \hline
28 & Dinas Peternakan, Kesehatan Hewan dan Perikanan & 31 \\ \hline
29 & Dinas Koperasi, Usaha Kecil, Menengah, Perdagangan dan Perindustrian & 10 \\ \hline
30 & Dinas Tenaga Kerja & 4 \\ \hline
31 & Dinas Pangan & 0 \\ \hline
32 & Dinas Pekerjaan Umum dan Tata Ruang & 6 \\ \hline
33 & Dinas Perumahan dan Kawasan Permukiman & 0 \\ \hline
34 & Dinas Perhubungan & 2 \\ \hline
35 & Dinas Lingkungan Hidup & 7 \\ \hline
36 & Badan Penanggulangan Bencana Daerah & 22 \\ \hline
37 & Dinas Sosial & 37 \\ \hline
38 & Dinas Penanaman Modal dan Pelayanan Terpadu Satu Pintu & 6 \\ \hline
39 & Dinas Pemberdayaan Masyarakat Desa & 10 \\ \hline
40 & Kecamatan Agrabinta & 8 \\ \hline
41 & Kecamatan Bojongpicung & 2 \\ \hline
42 & Kecamatan Cianjur & 13 \\ \hline
43 & Kecamatan Cibeber & 20 \\ \hline
44 & Kecamatan Cilaku & 10 \\ \hline
45 & Kecamatan Ciranjang & 2 \\ \hline
46 & Kecamatan Cugenang & 16 \\ \hline
47 & Kecamatan Cikalongkulon & 2 \\ \hline
48 & Kecamatan Campaka & 0 \\ \hline
49 & Kecamatan Cibinong & 5 \\ \hline
50 & Kecamatan Cidaun & 2 \\ \hline
51 & Kecamatan Campakamulya & 1 \\ \hline
52 & Kecamatan Cikadu & 1 \\ \hline
53 & Kecamatan Cijati & 1 \\ \hline
54 & Kecamatan Cipanas & 10 \\ \hline
55 & Kecamatan Gekbrong & 6 \\ \hline
56 & Kecamatan Haurwangi & 2 \\ \hline
57 & Kecamatan Karangtengah & 1 \\ \hline
58 & Kecamatan Kadupandak & 1 \\ \hline
59 & Kecamatan Leles & 3 \\ \hline
60 & Kecamatan Mande & 6 \\ \hline
61 & Kecamatan Naringgul & 6 \\ \hline
62 & Kecamatan Pacet & 20 \\ \hline
63 & Kecamatan Pagelaran & 1 \\ \hline
64 & Kecamatan Pasirkuda & 0 \\ \hline
65 & Kecamatan Sukaluyu & 4 \\ \hline
66 & Kecamatan Sukaresmi & 14 \\ \hline
67 & Kecamatan Sukanagara & 5 \\ \hline
68 & Kecamatan Sindangbarang & 1 \\ \hline
69 & Kecamatan Takokak & 6 \\ \hline
70 & Kecamatan Tanggeung & 1 \\ \hline
71 & Kecamatan Warungkondang & 0 \\ \hline
72 & Sekretariat Daerah & 5 \\ \hline
73 & RSUD Sayang & 1 \\ \hline
74 & RSUD Cimacan & 0 \\ \hline
75 & RSUD Pagelaran & 0 \\ \hline
76 & RSUD Sindangbarang & 0 \\ \hline
\rowcolor{lightgray}
\multicolumn{2}{|c|}{\textbf{Total}} & \textbf{651} \\ \hline

\end{longtable}
}

Sebagaimana disajikan pada Tabel~\ref{tab:rekap_jumlah_sop}, total SOP yang telah
disusun pada seluruh instansi di lingkungan Pemerintah Kabupaten Cianjur
berjumlah \textbf{651 SOP}. Jumlah tersebut tersebar pada berbagai perangkat
daerah dengan karakteristik dan kebutuhan layanan yang berbeda-beda. Instansi
yang memiliki ruang lingkup layanan luas dan kompleks cenderung memiliki jumlah
SOP yang lebih banyak, sementara instansi dengan fungsi yang lebih spesifik
memiliki jumlah SOP yang lebih terbatas.

Penyajian rekapitulasi SOP per instansi ini diharapkan dapat menjadi dasar bagi
pemerintah daerah dalam melakukan evaluasi capaian penyusunan SOP, pemetaan
kebutuhan penyempurnaan SOP, serta perencanaan penguatan penerapan SOP pada
masing-masing perangkat daerah.


\subsection{Rekapitulasi Status Verifikasi SOP}

Selain dilihat dari jumlah SOP per instansi, capaian kegiatan penyusunan dan
penataan SOP juga dianalisis berdasarkan status verifikasi SOP dalam sistem.
Rekapitulasi status verifikasi SOP memberikan gambaran mengenai posisi SOP pada
setiap tahapan proses penyusunan, mulai dari tahap konsep hingga SOP dinyatakan
aktif atau terbit.

{\footnotesize
\begin{longtable}{|
    >{\centering\arraybackslash}p{1cm}|
    p{7.5cm}|
    >{\centering\arraybackslash}p{3cm}|
}
\caption{Rekapitulasi Status Verifikasi SOP} \label{tab:rekap_status_verifikasi_sop} \\
\hline
\rowcolor{lightgray}\textbf{No} & \textbf{Status Verifikasi} & \textbf{Jumlah} \\
\hline
\endfirsthead

\hline
\rowcolor{lightgray}\textbf{No} & \textbf{Status Verifikasi} & \textbf{Jumlah} \\
\hline
\endhead

\hline
\endfoot

\hline
\endlastfoot

1 & Aktif & 2 \\ \hline
2 & Konsep & 581 \\ \hline
3 & Periksa & 61 \\ \hline
4 & Perbaikan & 3 \\ \hline
5 & Setuju & 5 \\ \hline
6 & Tolak & 1 \\ \hline
7 & Terbit & 0 \\ \hline
8 & Diganti & 0 \\ \hline
9 & Dicabut & 0 \\ \hline

\rowcolor{lightgray}
\multicolumn{2}{|c|}{\textbf{Total}} & \textbf{653} \\ \hline

\end{longtable}
}

Sebagaimana disajikan pada Tabel~\ref{tab:rekap_status_verifikasi_sop}, sebagian
besar SOP masih berada pada status \textit{konsep}, yaitu sebanyak \textbf{581 SOP}.
Hal ini menunjukkan bahwa proses penyusunan SOP masih terus berjalan dan
memerlukan tahapan lanjutan berupa pemeriksaan, perbaikan, dan persetujuan.
Sementara itu, terdapat SOP yang telah berada pada tahap \textit{periksa},
\textit{perbaikan}, dan \textit{setuju}, yang mencerminkan proses verifikasi yang
sedang berlangsung secara bertahap.

Adapun SOP yang telah mencapai status \textit{aktif} dan siap digunakan sebagai
pedoman kerja operasional masih berjumlah terbatas. Kondisi ini menunjukkan
bahwa fokus kegiatan pada periode pelaporan ini masih berada pada tahap
penyusunan dan penataan substansi SOP, yang selanjutnya akan dilanjutkan dengan
proses penetapan dan penerapan SOP secara lebih luas.

Rekapitulasi status verifikasi SOP ini memberikan gambaran yang jelas mengenai
progres penyusunan SOP secara keseluruhan serta menjadi dasar dalam perencanaan
tindak lanjut, khususnya percepatan proses verifikasi, penetapan, dan penerapan
SOP di lingkungan Pemerintah Kabupaten Cianjur.


