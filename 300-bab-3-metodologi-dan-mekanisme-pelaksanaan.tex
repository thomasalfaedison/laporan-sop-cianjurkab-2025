\phantomsection
\addcontentsline{toc}{chapter}{BAB III METODOLOGI DAN MEKANISME PELAKSANAAN}
\chapter*{BAB III \\ METODOLOGI DAN MEKANISME PELAKSANAAN}

\setcounter{chapter}{3} 
\setcounter{section}{0} 
\setcounter{figure}{0} 
\setcounter{table}{0}


Pelaksanaan kegiatan penyusunan dan penataan Standar Operasional Prosedur (SOP)
di lingkungan Pemerintah Kabupaten Cianjur dilaksanakan dengan mengacu pada
Pedoman Penyusunan Standar Operasional Prosedur Administrasi Pemerintahan
sebagaimana diatur dalam Peraturan Menteri Pendayagunaan Aparatur Negara dan
Reformasi Birokrasi Nomor 35 Tahun 2012.

Metode pelaksanaan disusun untuk memastikan bahwa proses penyusunan SOP
dilaksanakan secara sistematis, terstruktur, dan sesuai dengan kondisi serta
kebutuhan organisasi perangkat daerah. Metode ini mencakup penentuan lokasi
pelaksanaan, fokus penyusunan, sumber data, serta tahapan penyusunan SOP
yang meliputi pengumpulan data, pengorganisasian, dan pengembangan SOP.

\section{Lokasi Penyusunan}

Kegiatan penyusunan dan penataan Standar Operasional Prosedur (SOP)
dilaksanakan di lingkungan Pemerintah Kabupaten Cianjur, meliputi seluruh
perangkat daerah yang memiliki tugas dan fungsi dalam penyelenggaraan
administrasi pemerintahan dan pelayanan publik.

Pemilihan lokasi ini didasarkan pada kebutuhan Pemerintah Kabupaten Cianjur
untuk memiliki SOP yang terstandar, terdokumentasi dengan baik, serta sesuai
dengan karakteristik organisasi dan pelaksanaan tugas pada masing-masing
perangkat daerah.

\section{Fokus Penyusunan}

Fokus penyusunan SOP diarahkan pada proses-proses administrasi pemerintahan
dan pelayanan publik yang menjadi tugas dan fungsi perangkat daerah di
lingkungan Pemerintah Kabupaten Cianjur. Penyusunan SOP dilakukan untuk
membakukan proses kerja agar dapat dilaksanakan secara konsisten, efisien,
dan akuntabel.

Fokus penyusunan meliputi identifikasi kebutuhan SOP, penataan SOP yang telah
ada, serta penyusunan SOP baru sesuai dengan perubahan kebijakan, struktur
organisasi, dan kebutuhan pelayanan masyarakat.

\subsection{Sumber Data}

Sumber data yang digunakan dalam kegiatan penyusunan SOP terdiri atas data
primer dan data sekunder sebagai berikut.

\subsubsection*{Data Primer}

Data primer merupakan data yang diperoleh secara langsung dari perangkat
daerah di lingkungan Pemerintah Kabupaten Cianjur. Data primer diperoleh
melalui:
\begin{itemize}
    \item Wawancara dan diskusi dengan pejabat dan aparatur pelaksana
    yang terlibat langsung dalam proses administrasi pemerintahan dan
    pelayanan publik;
    \item Klarifikasi dan verifikasi kondisi aktual pelaksanaan tugas dan fungsi
    pada masing-masing perangkat daerah;
    \item Hasil pembahasan dan asistensi dalam proses penyusunan SOP.
\end{itemize}

\subsubsection*{Data Sekunder}

Data sekunder merupakan data pendukung yang diperoleh dari dokumen resmi
dan regulasi yang relevan dengan penyusunan SOP, antara lain:
\begin{itemize}
    \item Peraturan perundang-undangan terkait penyelenggaraan administrasi
    pemerintahan dan pelayanan publik;
    \item Pedoman Penyusunan Standar Operasional Prosedur Administrasi
    Pemerintahan;
    \item Dokumen SOP yang telah dimiliki oleh perangkat daerah;
    \item Dokumen pendukung lainnya yang berkaitan dengan tugas dan fungsi
    organisasi.
\end{itemize}

\section{Metode Penyusunan}

Metode penyusunan SOP dilaksanakan dengan mengacu pada siklus penyusunan
SOP Administrasi Pemerintahan, yang meliputi tahapan pengumpulan data,
pengorganisasian, dan pengembangan SOP.

\begin{figure}[H]
    \centering
    \includegraphics[width=13.45cm]{image/03-tahapan-penyusunan.png}
    \caption{Rincian Tahapan Penyusunan SOP AP}
    \label{fig:bagan-alir}
\end{figure}

\subsection{Tahap Persiapan dan Perencanaan}

Tahap persiapan dan perencanaan merupakan tahapan awal dalam pelaksanaan
penyusunan dan penataan Standar Operasional Prosedur (SOP) di lingkungan
Pemerintah Kabupaten Cianjur. Tahap ini bertujuan untuk memastikan bahwa
penyusunan SOP dilaksanakan secara terarah, sistematis, dan sesuai dengan
ketentuan peraturan perundang-undangan.

Pada tahap ini dilakukan identifikasi awal terhadap tugas dan fungsi perangkat
daerah serta proses-proses administrasi pemerintahan dan pelayanan publik yang
memerlukan pembakuan prosedur. Identifikasi dilakukan dengan mengacu pada
dokumen perencanaan instansi, seperti visi dan misi, tujuan dan sasaran organisasi,
rencana strategis, serta rencana kerja perangkat daerah, sehingga diperoleh
gambaran menyeluruh mengenai aktivitas dan proses kerja yang dijalankan.

\subsubsection{Inventarisasi dan Identifikasi Kebutuhan SOP}

Inventarisasi dilakukan terhadap SOP yang telah dimiliki oleh perangkat daerah
untuk mengetahui tingkat ketersediaan, kelengkapan, dan kesesuaiannya dengan
ketentuan yang berlaku serta kondisi aktual pelaksanaan tugas. Kegiatan ini
bertujuan untuk mengidentifikasi SOP yang masih relevan, SOP yang perlu
disempurnakan, serta kebutuhan penyusunan SOP baru.

Identifikasi kebutuhan SOP dilakukan dengan mempertimbangkan perubahan
kebijakan, struktur organisasi, serta tuntutan peningkatan kualitas pelayanan
publik di lingkungan Pemerintah Kabupaten Cianjur.

\subsubsection{Pengumpulan Data}

Pengumpulan data dilakukan sebagai dasar dalam penyusunan SOP dan meliputi
pengumpulan data primer dan data sekunder. Data primer diperoleh melalui
wawancara dan diskusi dengan pejabat dan aparatur yang terlibat langsung dalam
pelaksanaan proses kerja, untuk memperoleh informasi mengenai tahapan kegiatan,
tujuan proses, risiko yang melekat, serta kendala yang dihadapi dalam pelaksanaan
tugas.

Data sekunder diperoleh dari dokumen resmi, antara lain peraturan perundang-
undangan, pedoman teknis penyusunan SOP, dokumen organisasi, rencana strategis,
serta SOP yang telah tersedia pada masing-masing perangkat daerah.

\subsubsection{Pengorganisasian Informasi}

Pengorganisasian informasi dilakukan setelah seluruh data dan informasi yang
berkaitan dengan proses kerja terkumpul. Pada tahap ini dilakukan penataan dan
pengelompokan informasi agar lebih sistematis, mudah dipahami, dan siap
digunakan dalam proses penyusunan SOP.

Pengorganisasian dilakukan berdasarkan proses atau aktivitas kerja, bukan
berdasarkan unit organisasi, sehingga SOP yang disusun mencerminkan alur kerja
yang utuh dan lintas fungsi. Pengelompokan informasi dilakukan secara sederhana
dan praktis agar mudah diimplementasikan dalam pelaksanaan tugas sehari-hari.

Dengan dilaksanakannya tahap persiapan dan perencanaan secara menyeluruh,
diharapkan penyusunan SOP di lingkungan Pemerintah Kabupaten Cianjur dapat
menghasilkan SOP yang sesuai dengan kebutuhan organisasi serta mendukung
terwujudnya tata kelola pemerintahan yang tertib, efektif, dan akuntabel.

\subsection{Tahap Pengembangan SOP}

Tahap pengembangan SOP merupakan tahapan inti dalam siklus penyusunan Standar
Operasional Prosedur Administrasi Pemerintahan sebagaimana diatur dalam
Peraturan Menteri Pendayagunaan Aparatur Negara dan Reformasi Birokrasi Nomor
35 Tahun 2012. Tahap ini dilaksanakan setelah seluruh kegiatan persiapan dan
penilaian kebutuhan SOP selesai dilakukan.

Tahap pengembangan SOP sekaligus merupakan tahap pelaksanaan kegiatan
penyusunan SOP. Pada tahap ini dilakukan perumusan SOP berdasarkan hasil
pengumpulan dan pengorganisasian data yang telah diperoleh pada tahap
sebelumnya. Data dan informasi tersebut dianalisis untuk memastikan bahwa SOP
yang disusun mencerminkan alur pelaksanaan tugas dan fungsi yang berlaku secara
nyata di lingkungan Pemerintah Kabupaten Cianjur.

Dalam tahap pengembangan, penyusunan SOP dilakukan dengan menguraikan setiap
prosedur ke dalam tahapan kegiatan yang jelas, sistematis, dan berurutan. Setiap
tahapan prosedur dirumuskan dengan memperhatikan tujuan kegiatan, pelaksana,
alur kerja, serta hasil yang diharapkan. Penyusunan SOP dilakukan dengan mengacu
pada ketentuan peraturan perundang-undangan yang berlaku dan kebijakan internal
pemerintah daerah, sehingga SOP yang dihasilkan memiliki kepastian hukum dan
dapat dipertanggungjawabkan.

Selain perumusan prosedur, tahap pengembangan juga mencakup kegiatan
pembahasan dan klarifikasi bersama perangkat daerah terkait. Pembahasan ini
bertujuan untuk memperoleh kesepahaman mengenai substansi SOP, memastikan
kejelasan setiap tahapan prosedur, serta menghindari terjadinya perbedaan
penafsiran dalam pelaksanaan tugas dan pelayanan publik.

Hasil dari tahap pengembangan SOP adalah tersusunnya dokumen Standar
Operasional Prosedur Administrasi Pemerintahan yang lengkap, jelas, dan terstandar
sesuai dengan pedoman penyusunan SOP, serta siap untuk diterapkan dalam
penyelenggaraan administrasi pemerintahan dan pelayanan publik.

\subsection{Tahap Penerapan SOP}

Tahap penerapan SOP merupakan tahapan lanjutan setelah SOP selesai disusun dan
ditetapkan. Tahap ini merupakan bagian dari siklus penyusunan SOP Administrasi
Pemerintahan sebagaimana diatur dalam PermenPAN-RB Nomor 35 Tahun 2012,
yang menekankan bahwa SOP yang telah disusun harus digunakan secara nyata
dalam pelaksanaan tugas dan fungsi aparatur.

Pada tahap penerapan, SOP yang telah disusun mulai diberlakukan dan dijadikan
sebagai pedoman kerja resmi bagi aparatur di lingkungan Pemerintah Kabupaten
Cianjur. Aparatur melaksanakan tugas dan pelayanan publik dengan mengacu pada
SOP yang telah ditetapkan, sehingga pelaksanaan pekerjaan dilakukan secara
tertib, konsisten, dan sesuai dengan prosedur yang baku.

Penerapan SOP bertujuan untuk memastikan bahwa seluruh proses administrasi
pemerintahan dan pelayanan publik dilaksanakan secara seragam dan terstandar,
serta untuk meningkatkan kepastian prosedur, efisiensi kerja, dan akuntabilitas
pelaksanaan tugas. Melalui penerapan SOP, diharapkan dapat terbangun budaya
kerja aparatur yang berorientasi pada kepastian prosedur dan kualitas pelayanan.

Dalam konteks Pemerintah Kabupaten Cianjur, tahap penerapan SOP juga diarahkan
untuk mendukung peningkatan kinerja organisasi perangkat daerah serta
peningkatan kualitas pelayanan kepada masyarakat, sesuai dengan prinsip tata
kelola pemerintahan yang baik.

\subsection{Tahap Monitoring dan Evaluasi SOP}

Tahap monitoring dan evaluasi SOP merupakan tahapan akhir dalam siklus
penyusunan SOP Administrasi Pemerintahan sebagaimana diatur dalam
PermenPAN-RB Nomor 35 Tahun 2012. Tahap ini dilaksanakan untuk menilai
efektivitas dan kesesuaian penerapan SOP yang telah ditetapkan.

Monitoring SOP dilakukan untuk memastikan bahwa SOP diterapkan secara
konsisten oleh aparatur sesuai dengan ketentuan yang telah ditetapkan. Kegiatan
monitoring mencakup pengamatan terhadap pelaksanaan prosedur, kepatuhan
aparatur, serta kesesuaian pelaksanaan tugas dengan SOP yang berlaku.

Evaluasi SOP dilakukan untuk mengidentifikasi kendala, hambatan, serta potensi
perbaikan dalam penerapan SOP. Evaluasi juga mempertimbangkan perubahan
kebijakan, perubahan struktur organisasi, serta dinamika penyelenggaraan
administrasi pemerintahan dan pelayanan publik di lingkungan Pemerintah
Kabupaten Cianjur.

Hasil monitoring dan evaluasi digunakan sebagai dasar dalam melakukan perbaikan
dan penyempurnaan SOP secara berkelanjutan. Dengan demikian, SOP yang
diterapkan tetap relevan, aplikatif, dan mampu mendukung peningkatan kualitas tata
kelola pemerintahan dan pelayanan publik secara berkesinambungan.
