\phantomsection
\addcontentsline{toc}{chapter}{BAB III METODOLOGI DAN MEKANISME PELAKSANAAN}
\chapter*{BAB III \\ METODOLOGI DAN MEKANISME PELAKSANAAN}

\setcounter{chapter}{3} 
\setcounter{section}{0} 
\setcounter{figure}{0} 
\setcounter{table}{0}

\section{Metodologi Pelaksanaan}

Metodologi pelaksanaan kegiatan penyusunan dan penataan Standar Operasional
Prosedur (SOP) disusun untuk memastikan bahwa proses pengumpulan data,
analisis, dan penyusunan rekomendasi berjalan secara komprehensif, terstruktur,
serta dapat dipertanggungjawabkan. Metodologi ini dirancang agar SOP yang
dihasilkan dapat diimplementasikan secara efektif oleh seluruh perangkat daerah.

Pendekatan metodologis yang digunakan merupakan kombinasi pendekatan
kualitatif dan kuantitatif. Pendekatan kualitatif dilakukan melalui analisis kondisi
eksisting unit layanan, capaian kinerja, serta kesenjangan antara standar
pelayanan yang seharusnya dengan praktik pelaksanaan di lapangan. Sementara
itu, pendekatan kuantitatif digunakan untuk mendukung analisis berbasis data,
khususnya yang berkaitan dengan capaian pelayanan dan persepsi masyarakat.

Kegiatan penyusunan SOP menggunakan instrumen yang disusun berdasarkan
prinsip-prinsip pelayanan publik, standar pelayanan, serta pedoman teknis
penyusunan SOP yang berlaku. Instrumen tersebut digunakan untuk memastikan
keseragaman pendekatan dan konsistensi hasil pada seluruh perangkat daerah.

Dalam pelaksanaannya, penyusunan SOP bersumber pada tiga jenis data utama,
yaitu sebagai berikut:
\begin{itemize}
    \item \textbf{Data administratif}, yang diperoleh dari dokumen resmi unit layanan,
    seperti standar pelayanan, SOP yang telah ada, maklumat pelayanan, Survei
    Kepuasan Masyarakat (SKM), serta dokumen pendukung lainnya.
    
    \item \textbf{Data lapangan}, yang diperoleh melalui wawancara, diskusi, dan
    verifikasi langsung terhadap kondisi aktual pelaksanaan pelayanan pada
    masing-masing unit kerja.
    
    \item \textbf{Data persepsi masyarakat}, yang diperoleh melalui hasil survei
    kepuasan masyarakat atau masukan pengguna layanan yang relevan dengan
    jenis pelayanan yang diselenggarakan.
\end{itemize}

Metode pengumpulan data yang digunakan dalam kegiatan ini meliputi:
\begin{itemize}
    \item Penelaahan dokumen kebijakan, regulasi, dan standar pelayanan.
    \item Diskusi dan konsultasi teknis dengan perangkat daerah terkait.
    \item Workshop dan desk pemetaan penyusunan SOP di lingkungan Pemerintah
    Kabupaten Cianjur.
\end{itemize}

\section{Mekanisme Pelaksanaan}

Mekanisme pelaksanaan kegiatan penyusunan dan penataan SOP dilaksanakan
secara bertahap dan berkesinambungan untuk memastikan ketercapaian tujuan
kegiatan. Tahapan pelaksanaan disusun secara berurutan sebagai berikut:

\begin{enumerate}
    \item \textbf{Tahap Persiapan}

    Tahap persiapan meliputi perencanaan kegiatan, pembentukan dan penyiapan
    tim pelaksana, serta penyusunan rencana kerja dan jadwal pelaksanaan.
    Pada tahap ini juga dilakukan koordinasi awal dengan perangkat daerah
    untuk menyamakan pemahaman mengenai tujuan, ruang lingkup, dan
    mekanisme penyusunan SOP.

    \item \textbf{Tahap Pengumpulan Data}

    Tahap pengumpulan data dilaksanakan melalui inventarisasi SOP yang telah
    ada, pengumpulan dokumen pendukung, serta penggalian informasi terkait
    pelaksanaan tugas dan fungsi perangkat daerah. Data yang dikumpulkan
    menjadi dasar dalam proses analisis dan penyusunan SOP.

    \item \textbf{Tahap Pengolahan dan Analisis Data}

    Tahap ini dilakukan untuk mengolah dan menganalisis data yang telah
    dikumpulkan. Analisis difokuskan pada kesesuaian proses kerja dengan
    ketentuan peraturan perundang-undangan, efektivitas pelaksanaan, serta
    identifikasi kebutuhan penyusunan atau penyempurnaan SOP.

    \item \textbf{Tahap Penyusunan Laporan dan Rekomendasi}

    Tahap penyusunan laporan dan rekomendasi dilakukan dengan merumuskan
    hasil analisis ke dalam dokumen SOP yang terstandar. Pada tahap ini juga
    disusun rekomendasi perbaikan proses kerja untuk meningkatkan kualitas
    penyelenggaraan pelayanan dan pelaksanaan tugas pemerintahan.

    \item \textbf{Tahap Pemantauan Tindak Lanjut}

    Tahap pemantauan tindak lanjut dilaksanakan untuk memastikan bahwa SOP
    yang telah disusun dapat diterapkan secara konsisten oleh perangkat daerah.
    Pemantauan dilakukan sebagai bagian dari upaya perbaikan berkelanjutan
    dalam penyelenggaraan pemerintahan dan pelayanan publik.
\end{enumerate}
