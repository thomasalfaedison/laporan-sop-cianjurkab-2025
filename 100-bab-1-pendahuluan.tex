\chapter*{BAB I \\ PENDAHULUAN}
\addcontentsline{toc}{chapter}{BAB I PENDAHULUAN}

\setcounter{chapter}{1} 
\setcounter{section}{0} 
\setcounter{figure}{0} 
\setcounter{table}{0}

\section{Latar Belakang}

Dalam rangka mewujudkan tata kelola pemerintahan yang baik (good
governance), diperlukan adanya pedoman kerja yang jelas, terukur, dan
dapat dilaksanakan secara konsisten oleh seluruh aparatur pemerintahan.
Salah satu instrumen penting dalam pelaksanaan tugas dan fungsi
organisasi pemerintahan adalah Standar Operasional Prosedur (SOP).
SOP merupakan dokumen yang berisi langkah-langkah yang harus
dilakukan secara sistematis dan berurutan untuk menyelesaikan suatu
pekerjaan. Keberadaan SOP sangat penting untuk menciptakan
keseragaman dalam pelaksanaan tugas, menghindari kesalahan,
meningkatkan efisiensi dan efektivitas kinerja, serta memperkuat
akuntabilitas penyelenggaraan pemerintahan.
Namun dalam praktiknya, banyak unit kerja yang belum memiliki SOP
yang terdokumentasi secara baik, lengkap, dan sesuai dengan alur Standar
Operasional Prosedur yang ada. Selain itu, perkembangan regulasi dan
tuntutan pelayanan publik yang semakin dinamis memerlukan
penyusunan dan penyesuaian SOP secara berkala agar tetap relevan dan
aplikatif.
Sebagai bagian dari pelaksanaan reformasi birokrasi dan implementasi
Sistem Pemerintahan Berbasis Elektronik (SPBE), maka Pemerintah
Kabupaten Cianjur memandang perlu untuk melakukan penyusunan dan
penataan kembali SOP di seluruh perangkat daerah. Hal ini bertujuan
agar seluruh proses pelayanan dan pelaksanaan tugas pemerintahan dapat
berjalan secara transparan, terstandar, dan dapat dipertanggungjawabkan.
Dengan adanya penyusunan SOP ini, diharapkan setiap unit kerja memiliki
acuan yang seragam dalam menjalankan tugasnya, meningkatkan kualitas
layanan kepada masyarakat, serta mendukung pencapaian tujuan
organisasi secara optimal.

\section{Tujuan dan Sasaran}

\subsection{Tujuan Penyusunan Standar Operasional Prosedur (SOP)}

Penyusunan Standar Operasional Prosedur (SOP) bertujuan untuk:

\begin{enumerate}
    \item \textbf{Memberikan pedoman kerja yang baku dan terstandarisasi} bagi setiap unit kerja dalam melaksanakan tugas dan fungsi secara efisien, efektif, dan akuntabel.
    \item \textbf{Meningkatkan kualitas kinerja organisasi} melalui pelaksanaan tugas yang lebih sistematis, transparan, dan terdokumentasi.
    \item \textbf{Menjamin konsistensi pelaksanaan tugas dan pelayanan} kepada masyarakat melalui prosedur kerja yang jelas dan mudah dipahami.
    \item \textbf{Mendukung pengawasan dan evaluasi internal} dengan menyediakan dasar yang objektif terhadap pelaksanaan pekerjaan.
    \item \textbf{Mendorong budaya kerja profesional} dengan pembagian peran, tanggung jawab, dan alur kerja yang terdefinisi dengan baik.
    \item \textbf{Mendukung implementasi reformasi birokrasi dan digitalisasi tata kelola pemerintahan} melalui sistem informasi SOP.
\end{enumerate}

\subsection{Sasaran Penyusunan Standar Operasional Prosedur (SOP)}

Sasaran yang ingin dicapai dari penyusunan SOP adalah:

\begin{enumerate}
    \item \textbf{Tersusunnya dokumen SOP} yang sesuai dengan tugas dan fungsi unit kerja secara lengkap, sistematis, dan terdokumentasi.
    \item \textbf{Terlaksananya prosedur kerja yang seragam dan dapat diterapkan} oleh seluruh pegawai secara konsisten.
    \item \textbf{Tersedianya dasar dalam pelatihan dan pengenalan tugas bagi pegawai baru}, sehingga adaptasi kerja lebih cepat dan tepat.
    \item \textbf{Meningkatnya kepercayaan publik dan kepuasan layanan} karena proses pelayanan publik didasarkan pada SOP yang baku.
\end{enumerate}

\section{Landasan Hukum}

Adapun landasan hukum dalam Penyusunan Standar Operasional Prosedur
(SOP) ini adalah sebagai berikut:

\begin{enumerate}
    \item Permendagri Nomor 83 Tahun 2022 tentang Pedoman Penyusunan SOP di lingkungan pemerintah daerah.
    \item Undang – Undang Nomor 23 Tahun 2014 tentang Pemerintah Daerah, yang mengamanatkan penyelengaraan pemerintah daerah secara efektif, efisien, dan akuntabel.
    \item Peraturan Pemerintah Nomor 30 Tahun 2019 tentang penilian Kinerja Pegawai Negri Sipil, yang mewajibkan adanya indikator kinerja yang jelas dan terdokumentasi.
    \item Peraturan Menteri PAN dan RB Nomor Tahun 2012 tentang Pedoman Penyusunan Standar Operasional Administrasi Pemerintahan , yang menjadi rujukan teknis dalam pengembangan SOP.
    \item Permendagri Nomor 86 Tahun 2017 tentang Tata Cara Perencanaan, Pengendalian, dan Evaluasi Pembangunan Daerah, sebagain acuan integrasi perencanaan kinerja dengan sistem SOP.
\end{enumerate}

\section{Kondisi yang Diharapkan}

Kondisi yang diharapkan dari penyusunan Standar Operasional Prosedur
(SOP) adalah terwujudnya tata kelola pemerintahan yang lebih tertib,
efisien, dan akuntabel melalui pedoman kerja yang terdokumentasi,
sistematis, dan mudah diimplementasikan oleh seluruh perangkat daerah.
Penyusunan SOP diharapkan dapat memberikan manfaat nyata dalam
mendukung pelaksanaan tugas dan fungsi organisasi, dengan hasil akhir
berupa:

\begin{enumerate}
    \item \textbf{Tersusunnya dokumen SOP yang lengkap, jelas, dan sesuai} dengan masing-masing unit kerja di lingkungan Pemerintah Kabupaten Cianjur.
    \item \textbf{Terlaksananya kegiatan pelayanan publik secara konsisten}, berdasarkan langkah-langkah prosedural yang baku dan terstandarisasi.
    \item \textbf{Tersedianya acuan kerja yang dapat digunakan dalam pelatihan pegawai}, evaluasi kinerja, serta proses monitoring dan pengawasan.
    \item \textbf{Terciptanya keseragaman dan kepastian hukum dalam pelaksanaan tugas}, untuk menghindari multitafsir dan kesalahan prosedural.
    \item \textbf{Meningkatkan efisiensi waktu dan sumber daya}, karena setiap pekerjaan mengikuti alur dan tanggung jawab yang telah ditetapkan secara tepat.
    \item \textbf{Mendukung integrasi dengan sistem pemerintahan berbasis elektronik (SPBE)} sehingga mempermudah akses, pembaruan, dan pengawasan SOP secara digital.
    \item \textbf{Meningkatkan kepercayaan masyarakat terhadap layanan pemerintah}, karena prosedur pelayanan yang transparan dan profesional.
\end{enumerate}

Dengan tercapainya kondisi-kondisi tersebut, SOP diharapkan menjadi alat
manajemen operasional yang mendukung perbaikan berkelanjutan dalam
penyelenggaraan pemerintahan dan pelayanan publik.