\section{Rekapitulasi Layanan Perangkat Daerah}

Bagian ini menyajikan rekapitulasi Standar Operasional Prosedur (SOP) yang telah
disusun dan ditetapkan pada masing-masing instansi di lingkungan Pemerintah
Kabupaten Cianjur. Rekapitulasi SOP per instansi disusun sebagai bentuk
pertanggungjawaban pelaksanaan kegiatan penyusunan dan penataan SOP, sekaligus
sebagai gambaran capaian kegiatan pada tahap laporan akhir.

Rekapitulasi SOP disusun berdasarkan hasil inventarisasi dan penyusunan SOP yang
dilaksanakan pada setiap perangkat daerah, dengan memperhatikan kesesuaian
antara SOP yang disusun dengan tugas dan fungsi instansi, serta ketentuan
peraturan perundang-undangan yang berlaku. Setiap SOP yang direkapitulasi telah
melalui proses pengumpulan data, analisis, pembahasan, dan validasi bersama
instansi terkait, sehingga dapat digunakan sebagai pedoman kerja operasional.

Melalui rekapitulasi ini, dapat diketahui jumlah SOP yang telah disusun pada masing-
masing instansi, cakupan proses kerja yang telah terdokumentasi, serta tingkat
kesiapan perangkat daerah dalam menerapkan SOP secara konsisten. Rekapitulasi
ini juga memberikan gambaran mengenai pemerataan penyusunan SOP antar
instansi serta identifikasi instansi yang masih memerlukan penyempurnaan atau
penambahan SOP pada tahap selanjutnya.

Rekapitulasi SOP per instansi diharapkan dapat menjadi dasar dalam evaluasi
pelaksanaan SOP, perencanaan pembaruan dan penyempurnaan SOP secara
berkala, serta sebagai bahan pendukung dalam monitoring dan pengawasan
penyelenggaraan tugas dan pelayanan publik. Selain itu, rekapitulasi ini juga
mendukung integrasi SOP dengan Sistem Pemerintahan Berbasis Elektronik (SPBE)
sehingga pengelolaan SOP dapat dilakukan secara lebih terstruktur dan mudah
diakses.

Dengan adanya rekapitulasi SOP per instansi ini, diharapkan seluruh perangkat
daerah memiliki gambaran yang jelas mengenai ketersediaan dan cakupan SOP
yang dimiliki, serta dapat menjadikan SOP sebagai acuan utama dalam pelaksanaan
tugas dan fungsi secara tertib, efektif, dan akuntabel.

Sebagai tindak lanjut dari gambaran umum sebelumnya, berikut rekapitulasi jumlah layanan yang diselenggarakan oleh setiap perangkat daerah. Tabel berikut memberikan informasi mengenai jumlah layanan per perangkat daerah.

\begin{longtable}{
    |>{\centering\arraybackslash}p{0.8cm}
    |p{9.5cm}
    |>{\centering\arraybackslash}p{2cm}|
}
\caption{Jumlah SOP per Perangkat Daerah} \\
\hline
\rowcolor{gray!20}
\textbf{No} & \textbf{Perangkat Daerah} & \textbf{Jumlah SOP} \\
\hline
\endfirsthead

\hline
\rowcolor{gray!20}
\textbf{No} & \textbf{Perangkat Daerah} & \textbf{Jumlah SOP} \\
\hline
\endhead

\hline
\endfoot

\hline
\rowcolor{gray!20}
 & \textbf{Total} & \textbf{581} \\
\hline
\endlastfoot

1 & Sekretariat Daerah & 12 \\
2 & Sekretariat DPRD & 3 \\
3 & Inspektorat Daerah & 4 \\
4 & Dinas Pendidikan dan Kebudayaan & 11 \\
5 & Dinas Pariwisata, Kepemudaan dan Olahraga & 6 \\
6 & Dinas Kesehatan & 1 \\
7 & Dinas Perhubungan & 20 \\
8 & Dinas Komunikasi dan Informatika & 16 \\
9 & Dinas Pekerjaan Umum dan Penataan Ruang & 11 \\
10 & Dinas Perumahan, Kawasan Permukiman dan Pertanahan & 7 \\
11 & Dinas Koperasi, UMKM, Perdagangan dan Perindustrian & 5 \\
12 & Dinas Pertanian & 1 \\
13 & Dinas Peternakan dan Kesehatan Hewan & 2 \\
14 & Dinas Ketahanan Pangan & 4 \\
15 & Dinas Perikanan & 11 \\
16 & Dinas Kependudukan dan Pencatatan Sipil & 1 \\
17 & Dinas Sosial & 7 \\
18 & Dinas Tenaga Kerja, Transmigrasi, Energi dan Sumber Daya Mineral & 15 \\
19 & Dinas Lingkungan Hidup & 4 \\
20 & Dinas Pemberdayaan Masyarakat dan Desa & 1 \\
21 & Dinas Pengendalian Penduduk, KB, Pemberdayaan Perempuan, dan Perlindungan Anak & 4 \\
22 & Dinas Penanaman Modal dan Pelayanan Terpadu Satu Pintu & 30 \\
23 & Dinas Kearsipan dan Perpustakaan & 5 \\
24 & Satuan Polisi Pamong Praja dan Pemadam Kebakaran & 4 \\
25 & Badan Keuangan dan Aset Daerah & 5 \\
26 & Badan Pendapatan Daerah & 19 \\
27 & Badan Perencanaan Pembangunan, Penelitian dan Pengembangan Daerah & 13 \\
28 & Badan Kepegawaian dan Pengembangan Sumber Daya Manusia & 10 \\
29 & Badan Penanggulangan Bencana Daerah & 1 \\
30 & Badan Kesatuan Bangsa dan Politik & 2 \\
31 & Kecamatan Subang  & 11 \\
32 & Kecamatan Kalijati & 14 \\
33 & Kecamatan Cibogo & 11 \\
34 & Kecamatan Pagaden & 14 \\
35 & Kecamatan Binong & 11 \\
36 & Kecamatan Compreng & 14 \\
37 & Kecamatan Cipunagara & 10 \\
38 & Kecamatan Pamanukan & 10 \\
39 & Kecamatan Pusakanagara & 11 \\
40 & Kecamatan Legonkulon & 10 \\
41 & Kecamatan Ciasem & 10 \\
42 & Kecamatan Blanakan & 5 \\
43 & Kecamatan Patokbeusi & 10 \\
44 & Kecamatan Pabuaran & 12 \\
45 & Kecamatan Cipeundeuy & 14 \\
46 & Kecamatan Purwadadi & 14 \\
47 & Kecamatan Cikaum & 14 \\
48 & Kecamatan Cijambe & 7 \\
49 & Kecamatan Jalancagak & 14 \\
50 & Kecamatan Cisalak & 6 \\
51 & Kecamatan Tanjungsiang & 10 \\
52 & Kecamatan Sagalaherang & 14 \\
53 & Kecamatan Serangpanjang & 11 \\
54 & Kecamatan Sukasari & 5 \\
55 & Kecamatan Tambakdahan & 12 \\
56 & Kecamatan Kasomalang & 7 \\
57 & Kecamatan Dawuan & 14 \\
58 & Kecamatan Pagaden Barat & 10 \\
59 & Kecamatan Ciater & 10 \\
60 & Kecamatan Pusakajaya & 9 \\

\end{longtable}


Berdasarkan tabel jumlah layanan per perangkat daerah tersebut, dapat diketahui bahwa secara keseluruhan terdapat 581 layanan publik yang diselenggarakan oleh perangkat daerah di Kabupaten Subang. Jumlah ini menunjukkan bahwa pemerintah daerah telah menyediakan ragam layanan yang cukup luas untuk memenuhi kebutuhan masyarakat, pemerintah, maupun dunia usaha. Variasi jumlah layanan pada setiap perangkat daerah juga mencerminkan perbedaan fungsi, kewenangan, serta beban pelayanan yang diemban masing-masing organisasi.