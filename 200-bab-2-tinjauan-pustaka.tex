\chapter*{BAB II \\ TINJAUAN PUSTAKA}
\addcontentsline{toc}{chapter}{BAB II TINJAUAN PUSTAKA}

\setcounter{chapter}{2} 
\setcounter{section}{0} 
\setcounter{figure}{0} 
\setcounter{table}{0}

\section{Latar Belakang Penyusunan SOP Administrasi Pemerintahan}

Tujuan kebijakan Reformasi Birokrasi di Indonesia adalah untuk membangun profil
dan perilaku aparatur negara yang memiliki integritas, produktivitas, dan
bertanggung jawab serta memiliki kemampuan memberikan pelayanan yang prima
melalui perubahan pola pikir (\textit{mind set}) dan budaya kerja (\textit{culture set})
dalam sistem manajemen pemerintahan.

Reformasi Birokrasi mencakup delapan area perubahan utama pada instansi
pemerintah di pusat dan daerah, meliputi organisasi, tata laksana, peraturan
perundang-undangan, sumber daya manusia aparatur, pengawasan, akuntabilitas,
pelayanan publik, serta perubahan pola pikir dan budaya kerja aparatur.

Pada hakikatnya perubahan ketatalaksanaan diarahkan untuk melakukan penataan
tata laksana instansi pemerintah yang efektif dan efisien. Salah satu upaya penataan
tata laksana diwujudkan dalam bentuk penyusunan dan implementasi Standar
Operasional Prosedur Administrasi Pemerintahan (SOP AP) dalam pelaksanaan tugas
dan fungsi aparatur pemerintah.

\section{Pengertian Standar Operasional Prosedur Administrasi Pemerintahan}

Standar Operasional Prosedur adalah serangkaian instruksi tertulis yang dibakukan
mengenai berbagai proses penyelenggaraan aktivitas organisasi, bagaimana dan
kapan harus dilakukan, di mana dan oleh siapa dilakukan.

Standar Operasional Prosedur Administrasi Pemerintahan (SOP AP) adalah standar
operasional prosedur dari berbagai proses penyelenggaraan administrasi
pemerintahan yang sesuai dengan peraturan perundang-undangan yang berlaku.

Administrasi pemerintahan adalah pengelolaan proses pelaksanaan tugas dan fungsi
pemerintahan yang dijalankan oleh organisasi pemerintah.

\section{Manfaat Standar Operasional Prosedur Administrasi Pemerintahan}

Manfaat penyusunan dan penerapan SOP Administrasi Pemerintahan antara lain
sebagai standarisasi cara yang dilakukan aparatur dalam menyelesaikan pekerjaan
yang menjadi tugasnya, mengurangi tingkat kesalahan dan kelalaian yang mungkin
dilakukan oleh aparatur dalam melaksanakan tugas, serta meningkatkan efisiensi
dan efektivitas pelaksanaan tugas dan tanggung jawab aparatur dan organisasi
secara keseluruhan.

Selain itu, SOP Administrasi Pemerintahan dapat meningkatkan akuntabilitas
pelaksanaan tugas, menciptakan ukuran standar kinerja, menjamin konsistensi
pelayanan kepada masyarakat, serta memberikan perlindungan hukum bagi
aparatur dari kemungkinan tuntutan hukum karena tuduhan melakukan
penyimpangan.

\section{Prinsip Penyusunan dan Pelaksanaan SOP Administrasi Pemerintahan}

Prinsip penyusunan SOP Administrasi Pemerintahan meliputi kemudahan dan
kejelasan, efisiensi dan efektivitas, keselarasan, keterukuran, dinamis, berorientasi
pada pengguna atau pihak yang dilayani, kepatuhan hukum, serta kepastian hukum.

Prinsip pelaksanaan SOP Administrasi Pemerintahan meliputi konsistensi,
komitmen, perbaikan berkelanjutan, mengikat, seluruh unsur memiliki peran
penting, serta terdokumentasi dengan baik.

\section{Jenis Standar Operasional Prosedur Administrasi Pemerintahan}

Berdasarkan sifat kegiatannya, SOP Administrasi Pemerintahan terdiri atas SOP
Teknis dan SOP Administratif. SOP Teknis merupakan prosedur standar yang sangat
rinci dari kegiatan yang dilakukan oleh satu orang aparatur atau satu jabatan.
Sementara itu, SOP Administratif merupakan prosedur standar yang bersifat umum
dan tidak rinci dari kegiatan yang dilakukan oleh lebih dari satu aparatur atau
jabatan.

Berdasarkan cakupan dan besaran kegiatan, SOP Administrasi Pemerintahan
dibedakan menjadi SOP Makro dan SOP Mikro. SOP Makro mencakup beberapa SOP
Mikro yang membentuk satu rangkaian kegiatan, sedangkan SOP Mikro merupakan
bagian dari SOP Makro.

\section{Kedudukan SOP dalam Penataan Tata Laksana Pemerintahan}

Standar Operasional Prosedur Administrasi Pemerintahan merupakan salah satu
instrumen utama dalam penataan tata laksana pemerintahan. SOP AP berfungsi
sebagai pedoman kerja yang membakukan proses penyelenggaraan administrasi
pemerintahan agar dilaksanakan secara efektif, efisien, dan konsisten.

Dalam konteks Reformasi Birokrasi, SOP AP berperan sebagai jembatan antara
ketentuan peraturan perundang-undangan dengan praktik kerja aparatur. Dengan
adanya SOP, pelaksanaan tugas tidak bergantung pada individu, tetapi pada sistem
dan prosedur yang telah ditetapkan.

\section{SOP sebagai Instrumen Akuntabilitas dan Perlindungan Hukum}

SOP Administrasi Pemerintahan berfungsi sebagai instrumen akuntabilitas dalam
pelaksanaan tugas dan fungsi aparatur pemerintah. SOP memberikan standar
pelaksanaan kerja yang dapat digunakan sebagai dasar dalam pengukuran kinerja,
monitoring, serta evaluasi pelaksanaan tugas.

Selain itu, SOP AP juga memberikan perlindungan hukum bagi aparatur dalam
melaksanakan tugasnya. Aparatur yang melaksanakan pekerjaan sesuai dengan
SOP memiliki dasar yang jelas bahwa tindakan yang dilakukan telah sesuai dengan
ketentuan dan prosedur yang berlaku, sehingga meminimalkan risiko kesalahan
prosedural dan tuntutan hukum.

\section{Peran SOP dalam Penyelenggaraan Pelayanan Publik}

Dalam penyelenggaraan pelayanan publik, SOP Administrasi Pemerintahan
berperan penting dalam memberikan kepastian prosedur, waktu, dan hasil
pelayanan kepada masyarakat. SOP memastikan bahwa pelayanan dilaksanakan
secara konsisten dan sesuai dengan standar yang telah ditetapkan.

Penerapan SOP yang baik akan mendukung peningkatan kualitas pelayanan publik,
meningkatkan kepuasan masyarakat, serta memperkuat kepercayaan publik
terhadap kinerja pemerintah.

\section{Siklus Penyusunan SOP Administrasi Pemerintahan}

Penyusunan Standar Operasional Prosedur Administrasi Pemerintahan dilaksanakan
melalui suatu siklus yang terstruktur dan berkesinambungan. Siklus ini menunjukkan
bahwa SOP bukan merupakan dokumen yang bersifat statis, melainkan instrumen
manajemen yang harus senantiasa disesuaikan dengan perkembangan kebijakan,
organisasi, serta kebutuhan penyelenggaraan pemerintahan dan pelayanan publik.

Siklus penyusunan SOP Administrasi Pemerintahan meliputi tahapan persiapan,
penilaian kebutuhan SOP, pengembangan SOP, penerapan SOP, serta monitoring
dan evaluasi SOP. Setiap tahapan dalam siklus tersebut saling berkaitan dan
bertujuan untuk memastikan bahwa SOP yang disusun relevan, aplikatif, dan dapat
diterapkan secara konsisten.

Tahap persiapan dan penilaian kebutuhan SOP bertujuan untuk mengidentifikasi
proses administrasi pemerintahan yang memerlukan pembakuan prosedur serta
menentukan SOP yang perlu disusun atau disempurnakan. Tahap pengembangan
merupakan proses perumusan SOP secara sistematis berdasarkan hasil penilaian
kebutuhan dan ketentuan peraturan perundang-undangan yang berlaku.

Tahap penerapan SOP dilakukan dengan menggunakan SOP sebagai pedoman kerja
resmi dalam pelaksanaan tugas dan fungsi aparatur. Selanjutnya, monitoring dan
evaluasi SOP dilaksanakan untuk menilai tingkat kepatuhan, efektivitas, serta
kesesuaian penerapan SOP, sekaligus sebagai dasar dalam melakukan perbaikan
dan penyempurnaan SOP secara berkelanjutan.

Siklus penyusunan SOP Administrasi Pemerintahan menegaskan bahwa SOP
merupakan instrumen yang bersifat dinamis dan memerlukan peninjauan secara
berkala agar tetap mendukung peningkatan kualitas tata kelola pemerintahan dan
pelayanan publik.

\begin{figure}[H]
    \centering
    \includegraphics[width=10.45cm]{image/01-siklus-sop.png}
    \caption{Siklus Penyusunan SOP Administrasi Pemerintahan}
    \label{fig:bagan-alir}
\end{figure}