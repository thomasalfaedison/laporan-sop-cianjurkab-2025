\chapter*{BAB II \\ TINJAUAN PUSTAKA}
\addcontentsline{toc}{chapter}{BAB II TINJAUAN PUSTAKA}

\setcounter{chapter}{2} 
\setcounter{section}{0} 
\setcounter{figure}{0} 
\setcounter{table}{0}

\section{Pengertian Standar Operasional Prosedur Administrasi Pemerintahan}

Standar Operasional Prosedur adalah serangkaian instruksi tertulis yang dibakukan
mengenai berbagai proses penyelenggaraan aktivitas organisasi, bagaimana dan
kapan harus dilakukan, di mana dan oleh siapa dilakukan.

Standar Operasional Prosedur Administrasi Pemerintahan (SOP AP) adalah standar
operasional prosedur dari berbagai proses penyelenggaraan administrasi
pemerintahan yang sesuai dengan peraturan perundang-undangan yang berlaku.

Administrasi pemerintahan adalah pengelolaan proses pelaksanaan tugas dan fungsi
pemerintahan yang dijalankan oleh organisasi pemerintah.

\section{Manfaat Standar Operasional Prosedur Administrasi Pemerintahan}

Manfaat penyusunan dan penerapan SOP Administrasi Pemerintahan antara lain
sebagai standarisasi cara yang dilakukan aparatur dalam menyelesaikan pekerjaan
yang menjadi tugasnya, mengurangi tingkat kesalahan dan kelalaian yang mungkin
dilakukan oleh aparatur dalam melaksanakan tugas, serta meningkatkan efisiensi
dan efektivitas pelaksanaan tugas dan tanggung jawab aparatur dan organisasi
secara keseluruhan.

Selain itu, SOP Administrasi Pemerintahan dapat meningkatkan akuntabilitas
pelaksanaan tugas, menciptakan ukuran standar kinerja, menjamin konsistensi
pelayanan kepada masyarakat, serta memberikan perlindungan hukum bagi
aparatur dari kemungkinan tuntutan hukum karena tuduhan melakukan
penyimpangan.

\section{Prinsip Penyusunan dan Pelaksanaan SOP Administrasi Pemerintahan}

Prinsip penyusunan SOP Administrasi Pemerintahan meliputi kemudahan dan
kejelasan, efisiensi dan efektivitas, keselarasan, keterukuran, dinamis, berorientasi
pada pengguna atau pihak yang dilayani, kepatuhan hukum, serta kepastian hukum.

Prinsip pelaksanaan SOP Administrasi Pemerintahan meliputi konsistensi,
komitmen, perbaikan berkelanjutan, mengikat, seluruh unsur memiliki peran
penting, serta terdokumentasi dengan baik.

\section{Jenis Standar Operasional Prosedur Administrasi Pemerintahan}

Standar Operasional Prosedur Administrasi Pemerintahan dapat dibedakan ke dalam
beberapa jenis berdasarkan sifat kegiatan, cakupan dan besaran kegiatan,
cakupan dan kelengkapan kegiatan, serta cakupan dan jenis kegiatan.

\subsection{SOP Berdasarkan Sifat Kegiatan}

Berdasarkan sifat kegiatannya, Standar Operasional Prosedur Administrasi
Pemerintahan dibedakan menjadi dua jenis, yaitu:

\begin{enumerate}
    \item \textbf{SOP Teknis}

    SOP Teknis merupakan prosedur standar yang sangat rinci dari kegiatan yang
    dilakukan oleh satu orang aparatur atau satu jabatan. SOP Teknis memuat
    tahapan kegiatan yang bersifat teknis dan spesifik, sehingga hanya dapat
    dilaksanakan oleh aparatur tertentu sesuai dengan kompetensi dan
    kewenangannya.

    \item \textbf{SOP Administratif}

    SOP Administratif merupakan prosedur standar yang bersifat umum dan tidak
    rinci dari kegiatan yang dilakukan oleh lebih dari satu orang aparatur atau
    jabatan. SOP Administratif biasanya menggambarkan alur kegiatan yang
    melibatkan beberapa unit kerja atau jabatan dalam satu rangkaian proses
    administrasi pemerintahan.
\end{enumerate}

\subsection{SOP Berdasarkan Cakupan dan Besaran Kegiatan}

Berdasarkan cakupan dan besaran kegiatannya, Standar Operasional Prosedur
Administrasi Pemerintahan dibedakan menjadi dua jenis, yaitu:

\begin{enumerate}
    \item \textbf{SOP Makro}

    SOP Makro merupakan SOP yang menggambarkan rangkaian kegiatan secara
    umum dan mencakup beberapa SOP Mikro. SOP Makro digunakan untuk
    memberikan gambaran menyeluruh mengenai proses administrasi
    pemerintahan dalam suatu organisasi atau unit kerja.

    \item \textbf{SOP Mikro}

    SOP Mikro merupakan SOP yang menggambarkan kegiatan secara rinci dan
    merupakan bagian dari SOP Makro. SOP Mikro berfungsi sebagai pedoman
    operasional yang menjelaskan tahapan kegiatan secara detail dan teknis,
    sehingga dapat langsung digunakan oleh aparatur dalam melaksanakan
    tugasnya.
\end{enumerate}

\subsection{SOP Menurut Cakupan dan Kelengkapan Kegiatan}

Berdasarkan cakupan dan kelengkapan kegiatannya, Standar Operasional Prosedur
Administrasi Pemerintahan dibedakan menjadi dua jenis, yaitu:

\begin{enumerate}
    \item \textbf{SOP Lengkap}

    SOP Lengkap merupakan SOP yang memuat seluruh tahapan kegiatan secara
    utuh dan menyeluruh, mulai dari kegiatan awal hingga kegiatan akhir.
    SOP Lengkap digunakan sebagai pedoman kerja yang komprehensif dalam
    pelaksanaan suatu proses administrasi pemerintahan.

    \item \textbf{SOP Tidak Lengkap}

    SOP Tidak Lengkap merupakan SOP yang hanya memuat sebagian tahapan
    kegiatan dari suatu proses administrasi pemerintahan. SOP jenis ini biasanya
    digunakan sebagai bagian dari SOP yang lebih besar atau sebagai pelengkap
    dalam pelaksanaan SOP lainnya.
\end{enumerate}

\subsection{SOP Menurut Cakupan dan Jenis Kegiatan}

Berdasarkan cakupan dan jenis kegiatannya, Standar Operasional Prosedur
Administrasi Pemerintahan dibedakan menjadi dua jenis, yaitu:

\begin{enumerate}
    \item \textbf{SOP Internal}

    SOP Internal merupakan SOP yang mengatur pelaksanaan kegiatan yang
    dilakukan di dalam lingkungan organisasi atau unit kerja tertentu. SOP
    Internal tidak melibatkan pihak di luar organisasi dan digunakan sebagai
    pedoman kerja internal aparatur.

    \item \textbf{SOP Eksternal}

    SOP Eksternal merupakan SOP yang mengatur pelaksanaan kegiatan yang
    melibatkan pihak di luar organisasi pemerintah. SOP Eksternal biasanya
    berkaitan dengan pelayanan publik atau kegiatan lain yang berhubungan
    langsung dengan masyarakat atau instansi lain.

\end{enumerate}

\section{Format SOP}

Empat faktor yang dapat dijadikan dasar dalam penentuan format 
penyusunan SOP yang akan dipakai oleh suatu organisasi adalah:  
(1) berapa banyak keputusan yang akan dibuat dalam suatu prosedur;    
(2) berapa banyak langkah dan sub langkah yang diperlukan dalam suatu 
prosedur; (3) siapa yang dijadikan target sebagai pelaksana SOP; dan       
(4) apa tujuan yang ingin dicapai dalam pembuatan SOP ini. 
Format terbaik SOP adalah format yang sederhana dan dapat 
menyampaikan informasi yang dibutuhkan secara tepat serta memfasilitasi 
implementasi SOP secara konsisten sesuai dengan tujuan penyusunan 
SOP.

\subsection{Format Umum SOP}

\begin{enumerate}
    \item \textbf{Langkah sederhana (Simple Steps)}\\
    Simple steps adalah bentuk SOP yang paling sederhana. SOP ini  biasanya digunakan jika prosedur yang akan disusun hanya memuat  sedikit kegiatan dan memerlukan sedikit keputusan yang bersifat sederhana.
    \item \textbf{Tahapan berurutan (Hierarchical Steps)}\\
    Hierarchical Steps ini merupakan format pengembangan dari simple steps. Format ini digunakan jika prosedur yang disusun panjang, lebih dari 10 langkah dan membutuhkan informasi lebih detail, akan tetapi hanya memerlukan sedikit pengambilan keputusan.
    \item \textbf{Grafik (Graphic)}\\
    Format Grafik (graphic) dipilih, jika prosedur yang disusun menghendaki kegiatan yang panjang dan spesifik. Dalam format ini proses yang panjang tersebut dijabarkan ke dalam sub-sub proses yang lebih pendek yang hanya berisi beberapa langkah. Format ini juga bisa digunakan jika dalam menggambarkan prosedur diperlukan adanya suatu foto atau diagram.  Salah satu varian dari SOP format ini adalah SOP Format Annotated Picture (gambar yang diberi keterangan) yang biasanya ditujukan untuk pemohon atau pengguna jasa sebuah pelayanan. 
    \item \textbf{Diagram Alir (Flowcharts) }\\
    Flowcharts merupakan format yang biasa digunakan jika dalam SOP tersebut diperlukan pengambilan keputusan yang banyak (kompleks) dan membutuhkan opsi jawaban (alternatif jawaban) seperti: jawaban “ya” atau “tidak”, “lengkap” atau “tidak”, “benar” atau “salah”, dsb. yang akan mempengaruhi sub langkah berikutnya. Format ini juga menyediakan mekanisme yang mudah untuk diikuti dan dilaksanakan oleh para pelaksana (pegawai) melalui serangkaian langkah-langkah sebagai hasil dari keputusan yang telah diambil. Format SOP dalam bentuk flowcharts ini terdiri dari 2 (dua) jenis yaitu: 
    
    \begin{itemize}
        \item \textbf{Linear Flowcharts (Diagram Alir Linier)}\\
        Ciri utama dari format linear flowcharts ini adalah ada unsur kegiatan yang disatukan, yaitu: unsur kegiatan atau unsur pelaksananya dan menuliskan rumusan kegiatan secara singkat di dalam simbol yang dipakai. SOP format ini umumnya dipakai pada SOP yang bersifat teknis.

        \item \textbf{Branching Flowcharts (diagram alir bercabang)}\\
        Format Branching Flowcharts memiliki ciri utama dipisahkannya unsur pelaksana dalam kolom-kolom yang terpisah dari kolom kegiatan dan menggambarkan prosedur kegiatan dalam bentuk simbol yang dihubungkan secara bercabang-cabang.
        
    \end{itemize}

\end{enumerate}

\subsection{Format SOP AP}

\begin{enumerate}
    \item \textbf{Format Diagram Alir Bercabang (Branching Flowcharts) }\\
    Format yang dipergunakan dalam SOP AP adalah format diagram alir bercabang (branching flowcharts) dan tidak ada format lainnya yang dipakai. Hal ini diasumsikan bahwa prosedur pelaksanaan tugas dan fungsi instansi pemerintah termasuk di dalamnya Kementerian/Lembaga dan Pemerintah Daerah memuat kegiatan yang banyak (lebih dari sepuluh) dan memerlukan pengambilan keputusan yang banyak.

    \item \textbf{Menggunakan Flowcharts}\\
    Bagian Flowchart merupakan uraian mengenai langkah-langkah (prosedur) kegiatan beserta mutu baku dan keterangan yang diperlukan. Bagian Flowchart ini berupa flowcharts yang menjelaskan langkah-langkah kegiatan secara berurutan dan sistematis dari prosedur yang distandarkan, yang berisi: Nomor kegiatan; Uraian kegiatan yang berisi langkah-langkah (prosedur); Pelaksana yang merupakan pelaku (aktor) kegiatan; Mutu Baku yang berisi kelengkapan, waktu, output dan keterangan. Agar SOP AP ini terkait dengan kinerja, maka setiap aktivitas hendaknya mengidentifikasikan mutu baku tertentu, seperti: waktu yang diperlukan untuk menyelesaikan persyaratan/kelengkapan yang diperlukan (standar input) dan outputnya.
    Simbol yang digunakan dalam SOP AP hanya terdiri dari 5 (lima) simbol, yaitu: 4 (empat) simbol dasar flowcharts (Basic Symbol of Flowcharts) dan 1 (satu) simbol penghubung ganti halaman (Off-Page Conector). Kelima simbol yang dipergunakan tersebut adalah sebagai berikut:

    \begin{itemize}
        \item Simbol Kapsul/Terminator untuk mendeskripsikan kegiatan mulai dan berakhir; 
        \item Simbol Kotak/Process untuk mendeskripsikan proses atau kegiatan eksekusi; 
        \item Simbol Belah Ketupat/Decision untuk mendeskripsikan kegiatan pengambilan keputusan; 
        \item Simbol Anak Panah/Panah/Arrow untuk mendeskrpsikan arah kegiatan (arah proses kegiatan); 
        \item Simbol Segilima/Off-Page Connector  untuk mendeskripsikan hubungan antar simbol yang berbeda halaman. 
    \end{itemize}
\end{enumerate}

\begin{figure}[H]
    \centering
    \includegraphics[width=13.45cm]{image/02-format-sop.png}
    \caption{Contoh Format SOP Administrasi Pemerintahan}
    \label{fig:bagan-alir}
\end{figure}

\section{Kedudukan SOP dalam Penataan Tata Laksana Pemerintahan}

Standar Operasional Prosedur Administrasi Pemerintahan merupakan salah satu
instrumen utama dalam penataan tata laksana pemerintahan. SOP AP berfungsi
sebagai pedoman kerja yang membakukan proses penyelenggaraan administrasi
pemerintahan agar dilaksanakan secara efektif, efisien, dan konsisten.

Dalam konteks Reformasi Birokrasi, SOP AP berperan sebagai jembatan antara
ketentuan peraturan perundang-undangan dengan praktik kerja aparatur. Dengan
adanya SOP, pelaksanaan tugas tidak bergantung pada individu, tetapi pada sistem
dan prosedur yang telah ditetapkan.

\section{SOP sebagai Instrumen Akuntabilitas dan Perlindungan Hukum}

SOP Administrasi Pemerintahan berfungsi sebagai instrumen akuntabilitas dalam
pelaksanaan tugas dan fungsi aparatur pemerintah. SOP memberikan standar
pelaksanaan kerja yang dapat digunakan sebagai dasar dalam pengukuran kinerja,
monitoring, serta evaluasi pelaksanaan tugas.

Selain itu, SOP AP juga memberikan perlindungan hukum bagi aparatur dalam
melaksanakan tugasnya. Aparatur yang melaksanakan pekerjaan sesuai dengan
SOP memiliki dasar yang jelas bahwa tindakan yang dilakukan telah sesuai dengan
ketentuan dan prosedur yang berlaku, sehingga meminimalkan risiko kesalahan
prosedural dan tuntutan hukum.

\section{Peran SOP dalam Penyelenggaraan Pelayanan Publik}

Dalam penyelenggaraan pelayanan publik, SOP Administrasi Pemerintahan
berperan penting dalam memberikan kepastian prosedur, waktu, dan hasil
pelayanan kepada masyarakat. SOP memastikan bahwa pelayanan dilaksanakan
secara konsisten dan sesuai dengan standar yang telah ditetapkan.

Penerapan SOP yang baik akan mendukung peningkatan kualitas pelayanan publik,
meningkatkan kepuasan masyarakat, serta memperkuat kepercayaan publik
terhadap kinerja pemerintah.

\section{Siklus Penyusunan SOP Administrasi Pemerintahan}

Penyusunan Standar Operasional Prosedur Administrasi Pemerintahan dilaksanakan
melalui suatu siklus yang terstruktur dan berkesinambungan. Siklus ini menunjukkan
bahwa SOP bukan merupakan dokumen yang bersifat statis, melainkan instrumen
manajemen yang harus senantiasa disesuaikan dengan perkembangan kebijakan,
organisasi, serta kebutuhan penyelenggaraan pemerintahan dan pelayanan publik.

Siklus penyusunan SOP Administrasi Pemerintahan meliputi tahapan persiapan,
penilaian kebutuhan SOP, pengembangan SOP, penerapan SOP, serta monitoring
dan evaluasi SOP. Setiap tahapan dalam siklus tersebut saling berkaitan dan
bertujuan untuk memastikan bahwa SOP yang disusun relevan, aplikatif, dan dapat
diterapkan secara konsisten.

Tahap persiapan dan penilaian kebutuhan SOP bertujuan untuk mengidentifikasi
proses administrasi pemerintahan yang memerlukan pembakuan prosedur serta
menentukan SOP yang perlu disusun atau disempurnakan. Tahap pengembangan
merupakan proses perumusan SOP secara sistematis berdasarkan hasil penilaian
kebutuhan dan ketentuan peraturan perundang-undangan yang berlaku.

Tahap penerapan SOP dilakukan dengan menggunakan SOP sebagai pedoman kerja
resmi dalam pelaksanaan tugas dan fungsi aparatur. Selanjutnya, monitoring dan
evaluasi SOP dilaksanakan untuk menilai tingkat kepatuhan, efektivitas, serta
kesesuaian penerapan SOP, sekaligus sebagai dasar dalam melakukan perbaikan
dan penyempurnaan SOP secara berkelanjutan.

Siklus penyusunan SOP Administrasi Pemerintahan menegaskan bahwa SOP
merupakan instrumen yang bersifat dinamis dan memerlukan peninjauan secara
berkala agar tetap mendukung peningkatan kualitas tata kelola pemerintahan dan
pelayanan publik.

\begin{figure}[H]
    \centering
    \includegraphics[width=10.45cm]{image/01-siklus-sop.png}
    \caption{Siklus Penyusunan SOP Administrasi Pemerintahan}
    \label{fig:bagan-alir}
\end{figure}