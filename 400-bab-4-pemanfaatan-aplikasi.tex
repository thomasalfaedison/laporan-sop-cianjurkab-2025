\phantomsection
\addcontentsline{toc}{chapter}{BAB IV PEMANFAATAN APLIKASI PENYUSUN SOP}
\chapter*{BAB IV \\ PEMANFAATAN APLIKASI PENYUSUN SOP}

\setcounter{chapter}{4} 
\setcounter{section}{0} 
\setcounter{figure}{0} 
\setcounter{table}{0}

Dalam rangka mendukung penyusunan, penataan, serta pemantauan progres Standar
Operasional Prosedur (SOP) di lingkungan Pemerintah Kabupaten Cianjur, digunakan
aplikasi penyusun SOP yang terintegrasi. Pemanfaatan aplikasi ini dimaksudkan
untuk memastikan bahwa proses penyusunan SOP dilaksanakan secara sistematis,
terdokumentasi, dan mudah dipantau pada setiap tahapan pelaksanaannya.

Penyusunan SOP dilaksanakan dengan memanfaatkan aplikasi
penyusun SOP sebagai media utama kerja. Melalui aplikasi ini, proses pendampingan
tidak hanya dilakukan secara tatap muka atau manual, tetapi terintegrasi langsung
dalam sistem yang mendukung penyusunan SOP mulai dari tahap konsep hingga SOP
ditetapkan dan diberlakukan. Seluruh aktivitas penyusunan, pemeriksaan, perbaikan,
dan persetujuan SOP dilakukan melalui aplikasi sehingga setiap perkembangan dan
perubahan dapat terekam secara jelas dan terstruktur.

Aplikasi penyusun SOP berfungsi sebagai sarana pengelolaan SOP secara terpusat.
Melalui aplikasi ini, perangkat daerah dapat menyusun SOP sesuai dengan tugas
dan fungsinya, sementara tim pendamping dan pihak yang berwenang dapat
melakukan pemeriksaan, memberikan masukan, serta memantau progres
penyusunan SOP secara berkelanjutan. Dengan demikian, aplikasi ini berperan
sebagai alat bantu kerja sekaligus sarana monitoring dan pengendalian proses
penyusunan SOP.

\section{Pengguna dan Fitur Aplikasi Penyusun SOP}

Aplikasi penyusun SOP dirancang untuk digunakan oleh beberapa jenis pengguna
sesuai dengan peran dan kewenangannya dalam proses penyusunan SOP. Pembagian
pengguna dan fitur dalam aplikasi disusun untuk mendukung kelancaran proses
penyusunan SOP serta memastikan setiap tahapan dapat dilaksanakan secara
tertib dan terkontrol.

\subsection{Pengguna Aplikasi}

Pengguna aplikasi penyusun SOP dibagi ke dalam beberapa peran (role) sesuai
dengan kewenangan dan tanggung jawabnya dalam proses penyusunan dan
pengelolaan SOP. Pembagian peran ini bertujuan untuk memastikan bahwa setiap
tahapan penyusunan SOP dilaksanakan secara tertib, terkontrol, dan sesuai dengan
mekanisme yang telah ditetapkan.

Adapun peran pengguna dalam aplikasi penyusun SOP meliputi:

\begin{enumerate}
    \item \textbf{Pengguna Instansi}

    Pengguna instansi merupakan pengguna yang berasal dari perangkat daerah
    atau unit kerja penyusun SOP. Pengguna instansi memiliki kewenangan untuk
    melakukan penyusunan dan pengelolaan SOP pada tahap awal, antara lain:
    \begin{itemize}
        \item Melakukan penambahan data SOP;
        \item Melakukan perubahan atau pembaruan data SOP dengan status
        \textit{konsep};
        \item Menghapus data SOP yang masih berada pada status \textit{konsep}.
        
    \end{itemize}

    Pada peran ini, pengguna instansi bertanggung jawab memastikan bahwa
    substansi SOP yang disusun telah sesuai dengan tugas dan fungsi unit kerja
    sebelum diajukan ke tahap pemeriksaan dan persetujuan.

    \item \textbf{Pengguna Admin}

    Pengguna admin merupakan pengguna yang memiliki kewenangan pengelolaan
    aplikasi dan pengendalian proses penyusunan SOP secara keseluruhan.
    Pengguna admin memiliki hak akses yang lebih luas, meliputi:

    \begin{itemize}
        \item Melakukan penambahan, perubahan, dan penghapusan data SOP;
        \item Melakukan verifikasi terhadap SOP yang diajukan;
        \item Memberikan persetujuan atau penolakan terhadap SOP;
        \item Menerbitkan SOP yang telah disetujui agar dapat diberlakukan secara resmi.

    \end{itemize}

    Melalui peran admin, proses pengendalian mutu dan kepatuhan SOP terhadap
    ketentuan yang berlaku dapat dilakukan secara terpusat dan terkoordinasi.

\end{enumerate}

Pembagian peran pengguna ini mendukung keterlacakan proses penyusunan SOP,
memperjelas tanggung jawab pada setiap tahapan, serta meningkatkan
akuntabilitas dalam pengelolaan SOP di lingkungan Pemerintah Kabupaten Cianjur.

\subsection{Fitur Aplikasi Penyusun SOP}

Aplikasi penyusun SOP dilengkapi dengan fitur-fitur yang disusun berdasarkan alur
tahapan penyusunan SOP. Fitur-fitur tersebut dirancang untuk memudahkan
pengguna dalam mengelola SOP sesuai dengan status dan tahapan penyusunannya,
antara lain:

\begin{enumerate}
    \item \textbf{Dashboard}\\
    Menyajikan ringkasan informasi mengenai kondisi dan progres penyusunan SOP,
    termasuk jumlah SOP dan status SOP pada masing-masing tahapan.

    \item \textbf{SOP Unit}\\
    Digunakan untuk mengelola SOP yang disusun oleh masing-masing unit kerja
    atau perangkat daerah sesuai dengan tugas dan fungsinya.

    \item \textbf{SOP Umum}\\
    Memuat SOP yang bersifat lintas unit atau digunakan secara umum oleh lebih
    dari satu perangkat daerah.

    \item \textbf{Konsep}\\
    Berisi SOP yang masih berada pada tahap penyusunan awal dan masih dapat
    dilakukan penyempurnaan substansi.

    \item \textbf{Periksa}\\
    Digunakan untuk mengelola SOP yang sedang dalam tahap pemeriksaan guna
    memastikan kesesuaian dengan ketentuan yang berlaku.

    \item \textbf{Perbaikan}\\
    Memuat SOP yang memerlukan revisi atau penyempurnaan berdasarkan hasil
    pemeriksaan.

    \item \textbf{Setuju}\\
    Berisi SOP yang telah memperoleh persetujuan dan siap untuk ditetapkan.

    \item \textbf{Aktif}\\  
    Memuat SOP yang telah ditetapkan dan diberlakukan sebagai pedoman kerja
    resmi dalam pelaksanaan tugas dan pelayanan.

    \item \textbf{Arsip}\\  
    Digunakan untuk menyimpan SOP yang sudah tidak berlaku atau telah
    digantikan dengan SOP yang baru.
    
\end{enumerate}

\section{Alur Penyusun SOP di Aplikasi}

\begin{enumerate}
    \item \textbf{Dashboard}\\
    \begin{figure}[H]
        \centering
        \includegraphics[width=12.45cm]{image/app/01-dashboard.png}
        \caption{Dashboard SOP}
        \label{fig:bagan-alir}
    \end{figure}

    Pada Menu Dashboard SOP, ditampilkan informasi jumlah Total SOP, jumlah SOP dengan Status Konsep, jumlah SOP dengan Status Verifikasi, jumlah SOP dengan Status SOP Aktif, dan jumlah SOP pada Perangkat Daerah atau Instansi yang login.

    \item \textbf{Penambahan Data SOP}\\
    \begin{figure}[H]
        \centering
        \includegraphics[width=12.45cm]{image/app/02-sop-unit.png}
        \caption{SOP Unit}
        \label{fig:bagan-alir}
    \end{figure}

    Pada Menu SOP, User dapat Export Data SOP Unit, Filter Data SOP Unit, dan lihat detail data SOP Unit. Selain itu, dapat menambahkan data SOP dengan mengisi informasi pada form tambah berikut.
    \begin{figure}[H]
        \centering
        \includegraphics[width=11.25cm]{image/app/03-form-sop-unit.png}
        \caption{Form Tambah SOP}
        \label{fig:bagan-alir}
    \end{figure}

    Pada form tersebut, user akan diminta untuk memasukkan informasi Judul SOP, Nomor SOP, Tanggal Pembuatan, Tanggal Revisi, Tanggal Efektif, Kata Kunci, memilih SOP Level (Pilih Unit atau Umum), memilih Status Arsip (Ya atau Tidak), dan pilih Lintas Fungsi.

    \item \textbf{Melengkapi Data SOP}\\
    Setelah menyimpan data SOP tersebut, user perlu melengkapi informasi data SOP yang dibuat pada halaman detail SOP sebelum dikirimkan ke Verifikator.
    \begin{figure}[H]
        \centering
        \includegraphics[width=12.45cm]{image/app/09-informasi-detail-sop.png}
        \caption{Informasi Detail SOP}
        \label{fig:bagan-alir}
    \end{figure}

    Informasi yang perlu user lengkapi diantaranya tambah Dasar Hukum, Tambah Kualifikasi Pelaksana, Tambah Keterkaitan SOP, Tambah Peralatan/Perlengkapan, Tambah Peringatan, Tambah Pencatatan \& Pendataan, Tambah Pengesahan, tambah Pelaksana, dan Tambah Aktivitas. 

    Setelah user melengkapi informasi tersebut, maka dibagian bawah akan muncul Diagram SOP sesuai dengan informasi yang dimasukkan.
    \begin{figure}[H]
        \centering
        \includegraphics[width=13.45cm]{image/app/10-contoh-diagram.png}
        \caption{Contoh Diagram SOP}
        \label{fig:bagan-alir}
    \end{figure}
    
    \item \textbf{Kirim Data SOP}\\
    Setelah melengkapi data SOP dengan status Konsep, selanjutnya user dapat mengirimkan data SOP tersebut ke User Verifikator untuk dilakukan pemeriksaan. Untuk mengirimkan data tersebut, klik icon kirim pada halaman menu SOP seperti berikut.
    \begin{figure}[H]
        \centering
        \includegraphics[width=12.45cm]{image/app/04-kirim-konsep.png}
        \caption{Kirim Dokumen Konsep}
        \label{fig:bagan-alir}
    \end{figure}

    Setelah mengirimkan Data SOP maka data tersebut akan tampil di menu Periksa dan akan diperiksa oleh User Verifikator terkait.

    \begin{figure}[H]
        \centering
        \includegraphics[width=12.45cm]{image/app/05-periksa.png}
        \caption{Halaman Periksa Data}
        \label{fig:bagan-alir}
    \end{figure}

    Pada Menu Periksa, ada beberapa fitur yang user dapat gunakan seperti Export Data Periksa, Filter Data Periksa, dan lihat detail data Periksa. Data tersebut akan diperiksa oleh User yang memiliki akses untuk memeriksa data.

    \item \textbf{Verifikasi Data SOP}\\
    Untuk memeriksa data, user membuka halaman detail data dan akan tampil informasi data SOP serta dibawahnya terdapat Form untuk melakukan verifikasi data seperti berikut.
    \begin{figure}[H]
        \centering
        \includegraphics[width=12.45cm]{image/app/06-form-verifikasi.png}
        \caption{Form Verifikasi Data}
        \label{fig:bagan-alir}
    \end{figure}

    Pada form tersebut, user verifikator akan melakukan verifikasi terhadap komponen yang ada pada data yang dikirimkan. Dibagian paling bawah, user verifikator dapat menentukan status rekomendasi data SOP yang dikirimkan.
    \begin{figure}[H]
        \centering
        \includegraphics[width=12.45cm]{image/app/07-rekomendasi.png}
        \caption{Menentukan Rekomendasi}
        \label{fig:bagan-alir}
    \end{figure}

    Jika user verifikator memilih "Dapat Disahkan" maka Status Verifikasi akan berubah menjadi Setuju dan tampil pada menu Setuju. 
    
    \item \textbf{Melengkapi Dokumen SOP}\\
    Pada menu Setuju, User dapat melakukan Upload Dokumen SOP pada halaman detail SOP dengan klik Tombol "Unggah Dokumen" dan akan tampil form Unggah Dokumen. 
    \begin{figure}[H]
        \centering
        \includegraphics[width=12.45cm]{image/app/08-tombol-periksa.png}
        \caption{Upload Dokumen}
        \label{fig:bagan-alir}
    \end{figure}

    \item \textbf{Penerbitan SOP}\\
    Untuk menerbitkan SOP yang telah disetujui dan dilengkapi dokumennya, User Admin klik icon ceklis pada SOP yang akan diterbitkan seperti berikut.
    \begin{figure}[H]
        \centering
        \includegraphics[width=12.45cm]{image/app/11-icon-terbit.png}
        \caption{Penerbitan SOP}
        \label{fig:bagan-alir}
    \end{figure}

    User Admin juga bisa menerbitkan melalui halaman detail SOP. Pada halaman tersebut terdapat tombol "Terbitkan SOP", saat diklik maka SOP tersebut akan diterbitkan.
    \begin{figure}[H]
        \centering
        \includegraphics[width=12.45cm]{image/app/12-tombol-terbit.png}
        \caption{Tombol Terbitkan SOP}
        \label{fig:bagan-alir}
    \end{figure}

    Setelah icon Ceklis atau Tombol "Terbitkan SOP" diklik maka akan ditampilkan Form untuk mengisi Nomor SOP dan Tanggal Efektif.
    \begin{figure}[H]
        \centering
        \includegraphics[width=12.45cm]{image/app/13-terbitkan.png}
        \caption{Form Penerbitan SOP}
        \label{fig:bagan-alir}
    \end{figure}

    Setelah melengkapi informasi yang diminta, klik tombol "Terbitkan SOP" dan SOP tersebut akan diterbitkan dan Status menjadi Terbit.

\end{enumerate}